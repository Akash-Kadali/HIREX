%-------------------------
% Resume in Latex
% Author : Jake Gutierrez
% Based off of: https://github.com/sb2nov/resume
% License : MIT
%------------------------

\documentclass[letterpaper,10pt]{article}

\usepackage{latexsym}
\usepackage[empty]{fullpage}
\usepackage{titlesec}
\usepackage{marvosym}
\usepackage[usenames,dvipsnames]{color}
\usepackage{verbatim}
\usepackage{enumitem}
\usepackage[hidelinks]{hyperref}
\usepackage{fancyhdr}
\usepackage[english]{babel}
\usepackage{tabularx}
\usepackage[utf8]{inputenc}
\usepackage{fontawesome} % Add this to your preamble

\input{glyphtounicode}


%----------FONT OPTIONS----------
% sans-serif
% \usepackage[sfdefault]{FiraSans}
% \usepackage[sfdefault]{roboto}
% \usepackage[sfdefault]{noto-sans}
% \usepackage[default]{sourcesanspro}

% serif
% \usepackage{CormorantGaramond}
% \usepackage{charter}


\pagestyle{fancy}
\fancyhf{} % clear all header and footer fields
\fancyfoot{}
\renewcommand{\headrulewidth}{0pt}
\renewcommand{\footrulewidth}{0pt}

% Adjust margins
\addtolength{\oddsidemargin}{-0.5in}
\addtolength{\evensidemargin}{-0.5in}
\addtolength{\textwidth}{1in}
\addtolength{\topmargin}{-0.7in}
\addtolength{\textheight}{1.0in}

\urlstyle{same}

\raggedbottom
\raggedright
\setlength{\tabcolsep}{0in}

% Sections formatting
\titleformat{\section}{
  \vspace{-15pt}\scshape\raggedright\large
}{}{0em}{}[\color{black}\titlerule \vspace{-5pt}]

% Ensure that generate pdf is machine readable/ATS parsable
\pdfgentounicode=1

%-------------------------
% Custom commands
\newcommand{\resumeItem}[1]{
  \item\small{
    {#1 \vspace{-2pt}}
  }
}

\newcommand{\resumeSubheading}[4]{
  \vspace{-2pt}\item
    \begin{tabular*}{0.97\textwidth}[t]{l@{\extracolsep{\fill}}r}
      \textbf{#1} & #2 \\
      \textit{\small#3} & \textit{\small #4} \\
    \end{tabular*}\vspace{-7pt}
}

\newcommand{\resumeSubSubheading}[2]{
    \item
    \begin{tabular*}{0.97\textwidth}{l@{\extracolsep{\fill}}r}
      \textit{\small#1} & \textit{\small #2} \\
    \end{tabular*}\vspace{-7pt}
}

\newcommand{\resumeProjectHeading}[2]{
    \item
    \begin{tabular*}{0.97\textwidth}{l@{\extracolsep{\fill}}r}
      \small#1 & #2 \\
    \end{tabular*}\vspace{-7pt}
}

\newcommand{\resumeSubItem}[1]{\resumeItem{#1}\vspace{-7pt}}

\renewcommand\labelitemii{$\vcenter{\hbox{\tiny$\bullet$}}$}

\newcommand{\resumeSubHeadingListStart}{\begin{itemize}[leftmargin=0.15in, label={}]}
\newcommand{\resumeSubHeadingListEnd}{\end{itemize}}
\newcommand{\resumeItemListStart}{\begin{itemize}}
\newcommand{\resumeItemListEnd}{\end{itemize}\vspace{-5pt}}

%-------------------------------------------
%%%%%%  RESUME STARTS HERE  %%%%%%%%%%%%%%%%%%%%%%%%%%%%


\begin{document}

\begin{center}
    \textbf{\Huge \scshape Sri Akash Kadali} \\ \vspace{1pt}
    8417 48th Ave, College Park, MD, 20740 \\ \vspace{1pt}
     \textit{Availability: June 1st, 2026} \\ \vspace{1pt}
    \small 240-726-9356 $|$ \href{mailto:kadali18@umd.edu}{\underline{kadali18@umd.edu}} $|$ 
    \href{https://www.linkedin.com/in/sri-akash-kadali/}{\underline{https://www.linkedin.com/in/sri-akash-kadali/}} $|$
    \href{https://github.com/Akash-Kadali}{\underline{https://github.com/Akash-Kadali}}
\end{center}

%-----------EDUCATION-----------
\section{Education}
\resumeSubHeadingListStart
   \resumeSubheading
      {University of Maryland, College Park, United States \scriptsize{} }{CGPA: 3.55/4}
      {Master of Science in Applied Machine Learning}{August 2024 - May 2026}
      \resumeItemListStart
      \item \textbf{Relevant Coursework:}
      \resumeItemListEnd

      \resumeSubheading
      {Indian Institute of Information Technology, Vadodara, India \scriptsize{} }{CGPA: 8.78/10}
      {Bachelor of Technology in Computer Science and Engineering}{December 2020- June 2024}
      \resumeItemListStart
      \item \textbf{Relevant Coursework:}
\resumeItemListEnd
  \resumeSubHeadingListEnd

%-----------TECHNICAL SKILLS-----------
\section{Skills}
 \begin{itemize}[leftmargin=0.15in, label={}]
    \small{\item{
\textbf{Skill1:} a,b,c\\
\textbf{Skill2:}  d,e,f\\
\textbf{Skill3:} xx\\
\textbf{Skill4:}  xrs\\
\textbf{Skill5:} xxx\\
    }}
 \end{itemize}

%-----------EXPERIENCE-----------
\section{Experience}
  \resumeSubHeadingListStart
\resumeSubheading
  {Machine Learning Intern \href{https://github.com/Akash-Kadali/Supervised-Contrastive-Learning-with-Attention-Emotion-Synthesis-for-Implicit-Hate-Speech-Detection}{\faGithub}}{May 2023 -- December 2023}
  {Indian Institute of Technology, Indore}{Indore, India}

\resumeItemListStart
    \resumeItem{Designed and implemented a DeBERTa-based architecture augmented with emotion embeddings and word-level attention mechanisms using Bi-LSTM for implicit hate speech detection, achieving a 5\% improvement in F1-score over baseline models.}
    \resumeItem{Leveraged supervised contrastive learning to enhance feature representation, resulting in an 8\% increase in classification accuracy on the IHSate and IHC datasets.}
    \resumeItem{Developed emotion synthesis pipelines incorporating sentiment features using the NRC Lexicon, contributing to a 6\% boost in model precision.}
    \resumeItem{Applied extensive data augmentation techniques, including Replace Named Entities (RNE), Replace Scalar Adverbs (RSA), Back Translation (BT), and Generative Models (GM), leading to a 12\% reduction in data sparsity.}
\resumeItemListEnd

\resumeSubheading
  {Machine Learning Intern \href{https://github.com/Akash-Kadali/CaDT-Net-A-Cascaded-Deformable-Transformer-Network}{\faGithub}}{January 2024 -- June 2024}
  {National Institute of Technology, Jaipur}{Jaipur, India}


      \resumeItemListStart
        \resumeItem{Designed and deployed MaxViT-based models for histopathological image classification, achieving a 92\% classification accuracy on large-scale medical datasets.}
        \resumeItem{Engineered Cascaded Deformable Transformer Layers (CDTL) to improve feature dependency modeling by 20\%, optimizing AI workflows in medical imaging.}
        \resumeItem{Implemented Cascaded Deformable Self-Attention (CDSA) to enhance category-specific feature extraction by 18\% across 10,000+ annotated medical images.}
        \resumeItem{Reduced model convergence time by 25\% through integration of skip connections and deformable convolutions, enhancing computational efficiency for real-time applications.}
        \resumeItem{Developed classification pipelines for breast tumor analysis, achieving a 15\% reduction in misclassification rates and an F1-score of 0.91, supporting accurate clinical decision-making.}
    \resumeItemListEnd

\resumeSubheading
  {Machine Learning Intern \href{https://github.com/Akash-Kadali/A-Graph-Based-Framework-for-User-Level-Feature-Modeling-with-Contextual-Embeddings}{\faGithub}}{July 2024 -- December 2024}
  {Indian Institute of Technology, Indore}{Remote, USA}


      \resumeItemListStart
        \resumeItem{Designed and deployed MaxViT-based models for histopathological image classification, achieving a 92\% classification accuracy on large-scale medical datasets.}
        \resumeItem{Engineered Cascaded Deformable Transformer Layers (CDTL) to improve feature dependency modeling by 20\%, optimizing AI workflows in medical imaging.}
        \resumeItem{Implemented Cascaded Deformable Self-Attention (CDSA) to enhance category-specific feature extraction by 18\% across 10,000+ annotated medical images.}
        \resumeItem{Reduced model convergence time by 25\% through integration of skip connections and deformable convolutions, enhancing computational efficiency for real-time applications.}
        \resumeItem{Developed classification pipelines for breast tumor analysis, achieving a 15\% reduction in misclassification rates and an F1-score of 0.91, supporting accurate clinical decision-making.}
    \resumeItemListEnd

\resumeSubheading
  {Machine Learning Engineer}{May 2025 -- August 2025}
  {Ayar Labs}{Santa Clara, CA}
  \resumeItemListStart
    \resumeItem{Shipped \textbf{VISDOM – Pixels to Reports}, an end-to-end visual inspection platform for photonic laser-die components that takes raw images to validated CSV/Excel reports with operator review.}
    \resumeItem{Designed a multi-stage pipeline: \textbf{YOLOv8} for die/laser/ID cropping \(\rightarrow\) region slicing (\textit{Critical, Body, Laser}) \(\rightarrow\) \textbf{Transformer ensembles} (MaxViT/Swin/CoaT) for defect classification \(\rightarrow\) OCR verification \(\rightarrow\) reporting.}
    \resumeItem{Built dual OCR pathways: \textbf{YOLO-OCR} (detection+decoding) and a \textbf{CRNN+CTC} recognizer (32\(\times\)128 grayscale, greedy decode, charset \texttt{0123456789ABCDEFG}); reconciled predictions and flagged mismatches for human review.}
    \resumeItem{Achieved strong production accuracy on \textit{Critical} region: \(\ge\) \textbf{99\% overall accuracy} with \textbf{96\% recall} on minority classes; on a 3{,}500-image imbalanced test (3{,}200 good/300 bad) baseline misclassifications were \(\approx\)5; after rebalancing to stress recall, maintained \(\ge\)96\% accuracy.}
    \resumeItem{Tackled severe imbalance (\textit{e.g., Laser: 568 bad vs 33.2k good}) using \textbf{class weighting}, \textbf{Focal+CrossEntropy} hybrid, calibrated thresholds, and targeted augmentation; improved minority precision/recall while containing false positives.}
    \resumeItem{Showed why 3-class (\textit{Critical: Good, Facet Damage, Foreign Material}) dropped to \(\sim\)92\%; recovered to \(\sim\)\textbf{96\%} via focal loss, weighted loss, and defect-aware augmentation.}
    \resumeItem{Observed 70–80\% of errors repeat across runs; created \textbf{image-specific augmentation} (illumination, hue/brightness, micro-defect synthesis) to fix stubborn edge cases.}
    \resumeItem{Authored \textbf{laser-defect synthesis} tools: binary/normalized mask annotation, alpha-blending copy-paste, and experiments with \textbf{DFMGAN/CutPaste++} to boost scarce defect modes.}
    \resumeItem{Labeled/curated \textbf{100k+} region crops; enforced \textbf{no leakage} by shuffling and renaming across splits; evaluated multiple splits (80/10/10, 70/15/15, 60/20/20) and \textbf{K-Fold} for robustness.}
    \resumeItem{Standardized training across models (e.g., \textbf{MaxViT-512/CoaT/Swin}): label smoothing, freeze-then-unfreeze schedules, tuned LR, early stopping, and \textbf{confidence gating} at inference.}
    \resumeItem{Engineered \textbf{Modal} serverless GPU deployment (\textbf{T4}) with \textbf{volumes} for static weights, \textbf{FastAPI} endpoints for YOLO/CRNN/classifiers; optimized cold-start/concurrency and unit-tested health/readiness.}
    \resumeItem{Containerized pipeline with \textbf{Docker}; standardized \texttt{coat\_infer\_laser.py} as the canonical inference template (volumes, normalization option, thresholding, consistent logging).}
    \resumeItem{Built a \textbf{Flask} operator UI: folder selection (Body/Laser/Critical), paginated browsing, split-view for Laser halves, and \textbf{reclassification} workflow to correct human mislabels post-training.}
    \resumeItem{Delivered a \textbf{mismatch verification} web app: shows image + both OCR predictions; operator inputs final code; emits consolidated CSV for auditability.}
    \resumeItem{Automated \textbf{Drive-to-GPU-to-Drive} flow: OAuth to pull new images, run YOLO/OCR/classifiers, upload \textbf{final Excel/CSV} + thumbnails back to Google Drive; cleans temp files to control costs.}
    \resumeItem{Authored \textbf{drive\_infer.py} jobs for both OCR systems; decoded YOLO boxes left-to-right into sequences; compared to filename ground truth; logged PASS/FAIL and confusion patterns.}
    \resumeItem{Created \textbf{production config} hygiene: \texttt{.env} based secrets, route tokens, structured \textbf{logging}, pagination, and input validation; added simple \textbf{auth guards} on write endpoints.}
    \resumeItem{Implemented \textbf{lighting-condition} awareness (3 dark / 4 light regimes): added condition-specific augmentation and considered per-condition heads to reduce domain shift.}
    \resumeItem{Built \textbf{dataset tools}: extended \texttt{dataset\_tool.py} (mask co-transform, square/power-of-two packing, \texttt{.npy} mask saves) to prep GAN and classifier data consistently.}
    \resumeItem{Packaged demo builds and \textbf{operator-friendly} UX (keyboard shortcuts, zoom, undo stack, session save/resume); shipped \textbf{PWA-style} UI polish and Ayar-aligned theming.}
    \resumeItem{Led weekly \textbf{stakeholder demos} (hardware, operations, IT); translated failure modes into data or model fixes; proposed a \textbf{6-week} plan (extendable to 10) and executed milestones on time.}
    \resumeItem{Produced \textbf{board-ready slides} (metrics, pain\(\rightarrow\)solution narrative, risk \& cost controls) and documented the pipeline, metrics, and SOPs for handoff.}
    \resumeItem{Instituted \textbf{QA guardrails}: dataset versioning, seed control (\texttt{random\_state=42}), input checksuming, and reproducible eval scripts; wrote \textbf{ablation} notebooks for design decisions.}
    \resumeItem{Planned a daily \textbf{8:00 AM} orchestrator: detect new uploads \(\rightarrow\) crop \(\rightarrow\) tri-classify \(\rightarrow\) dual-OCR \(\rightarrow\) mismatch UI \(\rightarrow\) publish report; skip compute when no deltas.}
  \resumeItemListEnd


\resumeSubHeadingListEnd

%-----------LEADERSHIP AND ACHIEVEMENTS-----------
\section{Achievements and Leadership}
\begin{itemize}[leftmargin=0.15in, label={}]
    \item Published \emph{"CaDT-Net: Cascaded Deformable Transformer for Breast Cancer"} at ICONIP 2024, achieving 92\% accuracy in image classification using \textbf{Neural Networks}.
    \item Awarded \textbf{Gold Medal for Academic Excellence} as the top B.Tech graduate.
    \item Represented IIIT Vadodara at the \textbf{G20 Summit, India}, managing logistics for 50+ delegates.
    \item Solved 100+ LeetCode problems, focusing on Graphs, DP, and System Design.
\end{itemize}

%------------------------------------------
\end{document}
